\nonstopmode{}
\documentclass[a4paper]{book}
\usepackage[times,inconsolata,hyper]{Rd}
\usepackage{makeidx}
\makeatletter\@ifl@t@r\fmtversion{2018/04/01}{}{\usepackage[utf8]{inputenc}}\makeatother
% \usepackage{graphicx} % @USE GRAPHICX@
\makeindex{}
\begin{document}
\chapter*{}
\begin{center}
{\textbf{\huge Package `GrowthCurveME'}}
\par\bigskip{\large \today}
\end{center}
\ifthenelse{\boolean{Rd@use@hyper}}{\hypersetup{pdftitle = {GrowthCurveME: Mixed-Effects Modeling for Cellular Growth Data}}}{}
\ifthenelse{\boolean{Rd@use@hyper}}{\hypersetup{pdfauthor = {Anand Panigrahy}}}{}
\begin{description}
\raggedright{}
\item[Title]\AsIs{Mixed-Effects Modeling for Cellular Growth Data}
\item[Version]\AsIs{0.0.1}
\item[Description]\AsIs{This is an R package that provides user-friendly
wrappers to the saemix package for performing linear and non-linear
mixed-effects regression modeling for in-vitro growth-assay data to
account for clustering or longitudinal analysis via repeated
measurements. The package allows users to fit a variety of growth
models, including linear, exponential, logistic, and Gompertz
functions. For non-linear models, starting values are automatically
calculated. The package includes functions for summarizing models,
visualizing data and results, calculating doubling time and other key
statistics, and generating model diagnostic plots and residual summary
statistics. It also provides functions for generating
publication-ready summary tables for reports. Additionally, users can
fit linear and non-linear least-squares regression models if
clustering is not applicable.}
\item[License]\AsIs{GPL (>= 3)}
\item[Depends]\AsIs{R (>= 3.6.0)}
\item[Imports]\AsIs{dplyr (>= 1.1.4),
flextable (>= 0.9.6),
ggplot2 (>= 3.5.1),
knitr (>= 1.46),
magrittr (>= 2.0.3),
minpack.lm (>= 1.2-4),
moments (>= 0.14.1),
nlraa (>= 1.9.7),
patchwork (>= 1.2.0),
rlang (>= 1.1.3),
saemix (>= 3.3),
stringr (>= 1.5.1),
tibble (>= 3.2.1),
tidyr (>= 1.3.1),
viridis (>= 0.6.5)}
\item[Suggests]\AsIs{rmarkdown (>= 2.27),
testthat (>= 3.2.1.1)}
\item[Encoding]\AsIs{UTF-8}
\item[LazyData]\AsIs{true}
\item[Roxygen]\AsIs{list(markdown = TRUE)}
\item[RoxygenNote]\AsIs{7.3.2}
\item[Config/testthat/edition]\AsIs{3}
\end{description}
\Rdcontents{Contents}
\HeaderA{exponential\_mixed\_model}{Fit an exponential mixed-effects regression model}{exponential.Rul.mixed.Rul.model}
%
\begin{Description}
'exponential\_mixed\_model()' is a function utilized with the
\code{\LinkA{growth\_curve\_model\_fit}{growth.Rul.curve.Rul.model.Rul.fit}} function for fitting a mono-exponential
mixed-effects regression model to growth data utilizing the saemix package.
Starting values are derived from an initial least-squares model using
the \code{\LinkA{nlsLM}{nlsLM}} function.
\end{Description}
%
\begin{Usage}
\begin{verbatim}
exponential_mixed_model(
  data_frame,
  model_type = "mixed",
  fixed_rate = TRUE,
  num_chains = 1
)
\end{verbatim}
\end{Usage}
%
\begin{Arguments}
\begin{ldescription}
\item[\code{data\_frame}] A data frame object that at minimum contains three
variables:
\begin{itemize}

\item{} cluster - a character type variable used to specify how observations
are nested or grouped by a particular cluster. Note if using a
least-squares model, please fill in all values of cluster with a single
dummy character string, do NOT leave blank.
\item{} time - a numeric type variable used for measuring time such as
minutes, hours, or days
\item{} growth\_metric - a numeric type variable used for measuring growth
over time such as cell count or confluency

\end{itemize}


\item[\code{model\_type}] A character string specifying the type of regression
model to be used. If "mixed" a mixed-effects regression model will be used
with fixed and random effects to account for clustering. Defaults to "mixed".

\item[\code{fixed\_rate}] A logical value specifying whether the rate constant
of the function should be treated as a fixed effect (TRUE) or random
effect (FALSE). Defaults to TRUE

\item[\code{num\_chains}] A numeric value specifying the number of chains to run
in parallel in the MCMC algorithm of saemix. Defaults to 1.
\end{ldescription}
\end{Arguments}
%
\begin{Value}
Returns an exponential model object of class 'saemix' when a
mixed-effects model is specified or a model object of class 'nls' if a
least-squares model is specified.
\end{Value}
%
\begin{SeeAlso}
\code{\LinkA{growth\_curve\_model\_fit}{growth.Rul.curve.Rul.model.Rul.fit}}
\end{SeeAlso}
%
\begin{Examples}
\begin{ExampleCode}
# Load example data (exponential data from GrowthCurveME package)
data(exp_mixed_data)
# Fit an exponential mixed-effects growth model
exp_mixed_model <- growth_curve_model_fit(
  data_frame = exp_mixed_data,
  function_type = "exponential"
)
# Fit an exponential mixed-effected model using exponential_mixed_model()
exp_mixed_model <- exponential_mixed_model(data_frame = exp_mixed_data)
\end{ExampleCode}
\end{Examples}
\HeaderA{exp\_mixed\_data}{Sample exponential growth dataset}{exp.Rul.mixed.Rul.data}
\keyword{datasets}{exp\_mixed\_data}
%
\begin{Description}
A dataset containing the minimum required variables needed to input data
into the GrowthModelME package functions
\end{Description}
%
\begin{Usage}
\begin{verbatim}
exp_mixed_data
\end{verbatim}
\end{Usage}
%
\begin{Format}
A data frame with 240 rows and 3 variables:
\begin{description}

\item[cluster] A character type variable used to specify the clustering
of values by a particular metric. Note when selecting a least-squares
model instead of a mixed-effects model, do not leave this variable NA,
fill in this values for this variable with 1 repative dummy variable for
the package to run properly
\item[time] A numeric type variable for any measurement in time such as
minutes, hours, or days
\item[growth\_metric] A numeric type variable for measuring growth such as
confluency or cell count

\end{description}

\end{Format}
%
\begin{Source}
Created through simulation to serve as an example
\end{Source}
%
\begin{Examples}
\begin{ExampleCode}
data(exp_mixed_data)
\end{ExampleCode}
\end{Examples}
\HeaderA{gompertz\_mixed\_model}{Fit a Gompertz mixed-effects regression model}{gompertz.Rul.mixed.Rul.model}
%
\begin{Description}
'gompertz\_mixed\_model()' is a function utilized with the
\code{\LinkA{growth\_curve\_model\_fit}{growth.Rul.curve.Rul.model.Rul.fit}} function for fitting a Gompertz
mixed-effects regression model to growth data utilizing the saemix package.
Starting values are derived from an initial least-squares model using
the \code{\LinkA{nlsLM}{nlsLM}} function.
\end{Description}
%
\begin{Usage}
\begin{verbatim}
gompertz_mixed_model(
  data_frame,
  model_type = "mixed",
  fixed_rate = TRUE,
  num_chains = 1
)
\end{verbatim}
\end{Usage}
%
\begin{Arguments}
\begin{ldescription}
\item[\code{data\_frame}] A data frame object that at minimum contains three
variables:
\begin{itemize}

\item{} cluster - a character type variable used to specify how observations
are nested or grouped by a particular cluster. Note if using a
least-squares model, please fill in all values of cluster with a single
dummy character string, do NOT leave blank.
\item{} time - a numeric type variable used for measuring time such as
minutes, hours, or days
\item{} growth\_metric - a numeric type variable used for measuring growth
over time such as cell count or confluency

\end{itemize}


\item[\code{model\_type}] A character string specifying the type of regression
model to be used. If "mixed" a mixed-effects regression model will be used
with fixed and random effects to account for clustering. Defaults to "mixed".

\item[\code{fixed\_rate}] A logical value specifying whether the rate constant
of the function should be treated as a fixed effect (TRUE) or random
effect (FALSE). Defaults to TRUE

\item[\code{num\_chains}] A numeric value specifying the number of chains to run
in parallel in the MCMC algorithm of saemix. Defaults to 1.
\end{ldescription}
\end{Arguments}
%
\begin{Value}
Returns a Gompertz model object of class 'saemix' when a
mixed-effects model is specified or a model object of class 'nls' if a
least-squares model is specified.
\end{Value}
%
\begin{SeeAlso}
\code{\LinkA{growth\_curve\_model\_fit}{growth.Rul.curve.Rul.model.Rul.fit}}
\end{SeeAlso}
%
\begin{Examples}
\begin{ExampleCode}
# Load example data (Gompertz data from GrowthCurveME package)
data(gomp_mixed_data)
# Fit a Gompertz mixed-effects growth model
gomp_mixed_model <- growth_curve_model_fit(
  data_frame = gomp_mixed_data,
  function_type = "gompertz"
)
# Fit a Gompertz mixed-effected model using gompertz_mixed_model()
gomp_mixed_model <- gompertz_mixed_model(data_frame = gomp_mixed_data)
\end{ExampleCode}
\end{Examples}
\HeaderA{gomp\_mixed\_data}{Sample Gompertz growth dataset}{gomp.Rul.mixed.Rul.data}
\keyword{datasets}{gomp\_mixed\_data}
%
\begin{Description}
A dataset containing the minimum required variables needed to input data
into the GrowthModelME package functions
\end{Description}
%
\begin{Usage}
\begin{verbatim}
gomp_mixed_data
\end{verbatim}
\end{Usage}
%
\begin{Format}
A data frame with 975 rows and 3 variables:
\begin{description}

\item[cluster] A character type variable used to specify the clustering
of values by a particular metric. Note when selecting a least-squares
model instead of a mixed-effects model, do not leave this variable NA,
fill in this values for this variable with 1 repative dummy variable for
the package to run properly
\item[time] A numeric type variable for any measurement in time such as
minutes, hours, or days
\item[growth\_metric] A numeric type variable for measuring growth such
as confluency or cell count

\end{description}

\end{Format}
%
\begin{Source}
Created through simulation to serve as an example
\end{Source}
%
\begin{Examples}
\begin{ExampleCode}
data(log_mixed_data)
\end{ExampleCode}
\end{Examples}
\HeaderA{growth\_boostrap\_ci}{Create bootstrap estimates and 95\% confidence intervals for mixed-effects and/or least-squares models for each time-point}{growth.Rul.boostrap.Rul.ci}
%
\begin{Description}
This function leverages the \code{\LinkA{saemix.bootstrap}{saemix.bootstrap}} function
for mixed-effects models and the \code{\LinkA{predict\_nls}{predict.Rul.nls}} function
for least-squares models to compute bootstrapped 95\% confidence intervals for
each time-point for graphical purposes. Estimates of the fixed-effects
values are calculated based on the median (50th percentile) of the
simulated bootstrap data, with the 95\% confidence interval constructed
from the 2.5th and 97.5th percentiles.
\end{Description}
%
\begin{Usage}
\begin{verbatim}
growth_boostrap_ci(
  data_frame,
  growth_model_object,
  growth_model_summary_list,
  boot_n_sim = 200,
  mix_boot_method = "case"
)
\end{verbatim}
\end{Usage}
%
\begin{Arguments}
\begin{ldescription}
\item[\code{data\_frame}] A data frame object that at minimum contains three
variables:
\begin{itemize}

\item{} cluster - a character type variable used to specify how observations
are nested or grouped by a particular cluster. Note if using a
least-squares model, please fill in all values of cluster with a single
dummy character string, do NOT leave blank.
\item{} time - a numeric type variable used for measuring time such as
minutes, hours, or days
\item{} growth\_metric - a numeric type variable used for measuring growth
over time such as cell count or confluency

\end{itemize}


\item[\code{growth\_model\_object}] A saemix or nls type model object that is
created using \code{\LinkA{growth\_curve\_model\_fit}{growth.Rul.curve.Rul.model.Rul.fit}} when
return\_summary = FALSE

\item[\code{growth\_model\_summary\_list}] A list object created by the
\code{\LinkA{growth\_curve\_model\_fit}{growth.Rul.curve.Rul.model.Rul.fit}} function when return\_summary = TRUE.

\item[\code{boot\_n\_sim}] A numeric value specifying the number of bootstrap
simulations to be performed. See \code{\LinkA{saemix.bootstrap}{saemix.bootstrap}}
for mixed-effects models and \code{\LinkA{predict\_nls}{predict.Rul.nls}} for
least-squares models. Defaults to 200.

\item[\code{mix\_boot\_method}] For mixed-effects models, a character string
specifying the bootstrap algorithm to use. Options include "case",
"residual", "parametric" or "conditional". Defaults to "case". See
\code{\LinkA{saemix.bootstrap}{saemix.bootstrap}} for more details.
\end{ldescription}
\end{Arguments}
%
\begin{Value}
An appended version of growth\_model\_summary\_list with a fourth
data frame titled "boot\_sim"
\end{Value}
%
\begin{SeeAlso}
\code{\LinkA{growth\_curve\_model\_fit}{growth.Rul.curve.Rul.model.Rul.fit}}
\end{SeeAlso}
%
\begin{Examples}
\begin{ExampleCode}
# Fit an mixed-effects growth model to the data, return summary
# and bootstrap estimates (boot_n_sim set to 6 for speed)
exp_mixed_model_summary <- growth_curve_model_fit(
  data_frame = exp_mixed_data,
  function_type = "exponential",
  bootstrap_time = TRUE,
  boot_n_sim = 6
)
\end{ExampleCode}
\end{Examples}
\HeaderA{growth\_curve\_model\_fit}{Fit a growth function using mixed-effects regression modeling}{growth.Rul.curve.Rul.model.Rul.fit}
%
\begin{Description}
'growth\_curve\_model\_fit()' fits a mixed-effects model to a data frame based
on a user-defined function to account for clustering.
\end{Description}
%
\begin{Usage}
\begin{verbatim}
growth_curve_model_fit(
  data_frame,
  function_type = "exponential",
  model_type = "mixed",
  fixed_rate = TRUE,
  num_chains = 1,
  time_unit = "hours",
  return_summary = TRUE,
  bootstrap_time = FALSE,
  boot_n_sim = 200,
  mix_boot_method = "case"
)
\end{verbatim}
\end{Usage}
%
\begin{Arguments}
\begin{ldescription}
\item[\code{data\_frame}] A data frame object that at minimum contains three
variables:
\begin{itemize}

\item{} cluster - a character type variable used to specify how observations
are nested or grouped by a particular cluster. Note if using a
least-squares model, please fill in all values of cluster with a single
dummy character string, do NOT leave blank.
\item{} time - a numeric type variable used for measuring time such as
minutes, hours, or days
\item{} growth\_metric - a numeric type variable used for measuring growth
over time such as cell count or confluency

\end{itemize}


\item[\code{function\_type}] A character string specifying the function for
modeling the shape of the growth. Options include "exponential", "linear",
"logistic", or "gompertz".

\item[\code{model\_type}] A character string specifying the type of regression
model to be used. If "mixed" a mixed-effects regression model will be used
with fixed and random effects to account for clustering. Defaults to "mixed".

\item[\code{fixed\_rate}] A logical value specifying whether the rate constant
of the function should be treated as a fixed effect (TRUE) or random
effect (FALSE). Defaults to TRUE

\item[\code{num\_chains}] A numeric value specifying the number of chains to run
in parallel in the MCMC algorithm of saemix. Defaults to 1.

\item[\code{time\_unit}] A character string specifying the units in which time is
measured in. Defaults to "hours"

\item[\code{return\_summary}] A logical value specifying whether to return the
growth\_model\_summary\_list when TRUE (list object containing summarized data)
or the object model object when FALSE. Defaults to TRUE.

\item[\code{bootstrap\_time}] Logical value indicating whether to append
a data frame with bootstrap estimates and 95\% confidence intervals for each
time point. Defaults to FALSE. See \code{\LinkA{growth\_boostrap\_ci}{growth.Rul.boostrap.Rul.ci}}
for more details.

\item[\code{boot\_n\_sim}] A numeric value specifying the number of bootstrap
simulations to be performed. Defaults to 200.
See \code{\LinkA{growth\_boostrap\_ci}{growth.Rul.boostrap.Rul.ci}} for more details.

\item[\code{mix\_boot\_method}] For mixed-effects models ONLY, a character string
specifying the bootstrap algorithm to use. Options include "case",
"residual", "parametric" or "conditional". Defaults to "case".
See \code{\LinkA{growth\_boostrap\_ci}{growth.Rul.boostrap.Rul.ci}} for more details.
\end{ldescription}
\end{Arguments}
%
\begin{Value}
A list object with the following data frames within the list:
\begin{itemize}

\item{} model\_summary\_wide - a data frame with 1 row containing
key model estimates, doubling-time, and model metrics depending
on the model\_type and function\_type specified
\item{} model\_summary\_long - a data frame that is a long dataset version of
'model\_summary\_wide' that can be used to generate a table of the model
results (see function \code{\LinkA{growth\_model\_summary\_table}{growth.Rul.model.Rul.summary.Rul.table}})
\item{} model\_residual\_data - a data frame containing the original data
frame values as well as predicted values, residuals, and theoretical
quantiles of the residuals depending on the model\_type selected
(see functions \code{\LinkA{growth\_model\_residual\_plots}{growth.Rul.model.Rul.residual.Rul.plots}} and
\code{\LinkA{growth\_vs\_time\_plot}{growth.Rul.vs.Rul.time.Rul.plot}}
\item{} simulated\_data - A data frame containing the bootstrap estimates and
95\% confidence intervals for each time point.ONLY GENERATED WHEN
bootstrap\_time = TRUE

\end{itemize}

Note when return\_summary is FALSE, will return a model object of class
'saemix' when a mixed-effects model is specified or a model object of
class 'nls' if a least-squares model is specified.
\end{Value}
%
\begin{Examples}
\begin{ExampleCode}
# Load example data (exponential data)
data(exp_mixed_data)
# Fit an mixed-effects growth model to the data and return summary
exp_mixed_model_summary <- growth_curve_model_fit(
data_frame = exp_mixed_data,
function_type = "exponential")
# Create flextable object from the summary list object for documentation
exp_model_table <- growth_model_summary_table(
growth_model_summary_list = exp_mixed_model_summary)
exp_model_table
# Create growth vs time plot of data with fitted values (plot_type = 2),
# adjust aesthetics and parameters as desired
exp_growth_plot <- growth_vs_time_plot(
growth_model_summary_list = exp_mixed_model_summary,
plot_type = 2)
print(exp_growth_plot)
# Check residuals and model assumptions
residual_diag_plot <- growth_model_residual_plots(
growth_model_summary_list = exp_mixed_model_summary)
print(residual_diag_plot)
\end{ExampleCode}
\end{Examples}
\HeaderA{growth\_model\_residual\_plots}{Create residual diagnostic plots for growth model}{growth.Rul.model.Rul.residual.Rul.plots}
%
\begin{Description}
'growth\_model\_residual\_plots()' is function that generates residual
diagnostic plots and summary statistics for a growth model summary list
object produced by \code{\LinkA{growth\_curve\_model\_fit}{growth.Rul.curve.Rul.model.Rul.fit}}.
\end{Description}
%
\begin{Usage}
\begin{verbatim}
growth_model_residual_plots(
  growth_model_summary_list,
  residual_type = "cluster",
  weighted = TRUE
)
\end{verbatim}
\end{Usage}
%
\begin{Arguments}
\begin{ldescription}
\item[\code{growth\_model\_summary\_list}] A list object created by the
\code{\LinkA{growth\_curve\_model\_fit}{growth.Rul.curve.Rul.model.Rul.fit}} function.

\item[\code{residual\_type}] A character string specifying the type of residuals
to be displayed in the plot. Options include "population" for the
fixed-effects residuals for mixed-effects and least-squares models and
"cluster" for fixed and random-effects residuals for mixed-effects
regression models. Defaults to "cluster".

\item[\code{weighted}] A logical value, when TRUE displays weighted residuals for
mixed-effects models or Standardized residuals for least-squares models,
when FALSE displays the raw residuals for mixed-effects and least-squares
models. Defaults to "TRUE".
\end{ldescription}
\end{Arguments}
%
\begin{Value}
Returns a ggplot2 collage of model diagnostic plots with the
following plots displayed:
\begin{itemize}

\item{} Residual vs Fitted Values - a model diagnostic plot for assessing
the distribution of standardized residuals vs the model fitted values,
useful in detecting improper function specification, homogeneity of
variance, and outlier detection.
\item{} Q-Q Plot - a model diagnostic plot (quantile-quantile) plot for
comparing standardized residuals vs their theoretical quantiles,
useful in assessing normality assumptions and outlier detection.
\item{} Residual Density Plot - a model diagnostic showing the distribution
of standardized residuals (histogram) with a normal distribution curve
overlaid based on the residuals mean and standard deviation, useful in
assessing normality assumptions and skewness.
\item{} Residual Summary Statistics - a list of descriptive statistics of
the standardized residuals including: mean, median, minimum, maximum,
skewness, and kurtosis.

\end{itemize}

\end{Value}
%
\begin{SeeAlso}
\code{\LinkA{growth\_curve\_model\_fit}{growth.Rul.curve.Rul.model.Rul.fit}}
\end{SeeAlso}
%
\begin{Examples}
\begin{ExampleCode}
# Load example data (exponential data)
data(exp_mixed_data)
# Fit an mixed-effects growth model to the data and produce summary list
exp_mixed_model_summary <- growth_curve_model_fit(
data_frame = exp_mixed_data,
function_type = "exponential")
# Check residuals and model assumptions
residual_diag_plot <- growth_model_residual_plots(
  growth_model_summary_list = exp_mixed_model_summary)
print(residual_diag_plot)
\end{ExampleCode}
\end{Examples}
\HeaderA{growth\_model\_summary\_table}{Create a printable table of the summarized growth model result reporting}{growth.Rul.model.Rul.summary.Rul.table}
%
\begin{Description}
'growth\_model\_summary\_table()' creates a flextable object that can be used
for documentation or Rmarkdown reports from the list object created
by \code{\LinkA{growth\_curve\_model\_fit}{growth.Rul.curve.Rul.model.Rul.fit}}.
The 'model\_summary\_long' data frame from the list object is used to
generate the table.
\end{Description}
%
\begin{Usage}
\begin{verbatim}
growth_model_summary_table(
  growth_model_summary_list,
  font_name = "Albany AMT",
  font_size_header = 14,
  font_size_body = 12,
  use_knit_print = FALSE
)
\end{verbatim}
\end{Usage}
%
\begin{Arguments}
\begin{ldescription}
\item[\code{growth\_model\_summary\_list}] A list object created by the
\code{\LinkA{growth\_curve\_model\_fit}{growth.Rul.curve.Rul.model.Rul.fit}} function.

\item[\code{font\_name}] A character string specifying the name of the font to use
when rendering the table. Defaults to "Albany AMT".
See \code{\LinkA{font}{font}}.

\item[\code{font\_size\_header}] A numeric value specifying the size of the font
for the header of the table. Defaults to 14.
See \code{\LinkA{fontsize}{fontsize}}

\item[\code{font\_size\_body}] A numeric value specifying the size of the font
for the header of the table. Defaults to 12.
See \code{\LinkA{fontsize}{fontsize}}

\item[\code{use\_knit\_print}] A logical value to specify whether the flextable
should be printer using the \code{\LinkA{knit\_print}{knit.Rul.print}} function
instead of the flextable object being returned.
Defaults to FALSE.
\end{ldescription}
\end{Arguments}
%
\begin{Value}
A flextable object of the 'model\_summary\_long' data frame
\end{Value}
%
\begin{SeeAlso}
\code{\LinkA{growth\_curve\_model\_fit}{growth.Rul.curve.Rul.model.Rul.fit}}
\end{SeeAlso}
%
\begin{Examples}
\begin{ExampleCode}
# Load example data (exponential data)
data(exp_mixed_data)
# Fit an mixed-effects growth model to the data
exp_mixed_model_summary <- growth_curve_model_fit(
data_frame = exp_mixed_data,
function_type = "exponential")
# Create flextable object of the growth model results
exp_model_table <- growth_model_summary_table(
growth_model_summary_list = exp_mixed_model_summary)
# Print the table in the view pane
exp_model_table
\end{ExampleCode}
\end{Examples}
\HeaderA{growth\_vs\_time\_plot}{Generate growth vs time plots}{growth.Rul.vs.Rul.time.Rul.plot}
%
\begin{Description}
'growth\_vs\_time\_plot()' is a function that can be used to generate
different plots from a list object created by the
\code{\LinkA{growth\_curve\_model\_fit}{growth.Rul.curve.Rul.model.Rul.fit}} function.
Please refer to the documentation for the 'plot\_type' parameter for the
different plot options.
\end{Description}
%
\begin{Usage}
\begin{verbatim}
growth_vs_time_plot(
  growth_model_summary_list,
  plot_type = 2,
  growth_metric_name = "growth_metric",
  time_name = "time",
  cluster_name = "cluster",
  plot_title = "Growth vs Time",
  x_aix_breaks = ggplot2::waiver(),
  x_limits = c(NA, NA),
  n_x_axis_breaks = NULL,
  y_aix_breaks = ggplot2::waiver(),
  y_limits = c(NA, NA),
  n_y_axis_breaks = NULL,
  x_axis_text_size = 8,
  y_axis_text_size = 12,
  x_axis_title_size = 14,
  y_axis_title_size = 14,
  plot_title_size = 20,
  geom_point_size = 2,
  geom_line_width = 0.5,
  ci_plot_annoate_value = "double_time",
  annotate_value_text_size = 6
)
\end{verbatim}
\end{Usage}
%
\begin{Arguments}
\begin{ldescription}
\item[\code{growth\_model\_summary\_list}] A list object created by the
\code{\LinkA{growth\_curve\_model\_fit}{growth.Rul.curve.Rul.model.Rul.fit}} function.

\item[\code{plot\_type}] A numeric value used to specify the plot type to graph.
Values include 1, 2, 3, 4 with descriptions of each below (defaults to 2):
\begin{itemize}

\item{} 1 - A scatterplot of the growth\_metric vs time data where each
point is colored by cluster if applicable.
\item{} 2 - A scatterplot of the growth\_metric vs time data where each
point is colored by cluster if applicable and the model predicted values
are overlayed as line.
When a mixed-effect model summary list is in-putted, the predicted values
will be the ind\_fit\_value which accounts for both fixed and random effects.
When a least-squares model summary list is in-putted the predicted values
will be the fitted values accounting for fixed effects only (pop\_fit\_value).
\item{} 3 - A scatterplot version of plot\_type = 2 where each cluster is
separated into their own plot forming a matrix of growth\_metric vs time
plots by cluster.
\item{} 4 - A plot of bootstrapped estimates as a smooth line,
with corresponding bootstrapped 95\% confidence intervals as a
shaded region. NOTE: plot can only be generated if
bootstrap\_time was set to TRUE in \code{\LinkA{growth\_curve\_model\_fit}{growth.Rul.curve.Rul.model.Rul.fit}} when
growth\_model\_summary\_list list object was generated.

\end{itemize}


\item[\code{growth\_metric\_name}] A character string for specifying the name of
the growth metric (y-axis title) to be displayed on the plot.
Defaults to "growth\_metric".

\item[\code{time\_name}] A character string for specifying the name of the time
variable (x-axis title) to be displayed on the plot. Defaults to "time".

\item[\code{cluster\_name}] A character string for specifying the name of the
cluster variable (legend title) to be displayed on the plot.
Defaults to "cluster".

\item[\code{plot\_title}] A character string for specifying the title to be
displayed over the plot. Defaults to "Growth vs Time".

\item[\code{x\_aix\_breaks}] A numeric vector specifying manual numeric breaks.
Defaults to ggplot2::waiver(). See \code{\LinkA{scale\_x\_continuous}{scale.Rul.x.Rul.continuous}}.

\item[\code{x\_limits}] A numeric vector of length two providing limits for
the x-axis. Use NA to refer to the existing minimum or maximum.
Defaults to c(NA, NA). See \code{\LinkA{scale\_x\_continuous}{scale.Rul.x.Rul.continuous}}.

\item[\code{n\_x\_axis\_breaks}] An integer specifying the number of major breaks
for the x-axis. Defaults to NULL.
See \code{\LinkA{scale\_x\_continuous}{scale.Rul.x.Rul.continuous}}.

\item[\code{y\_aix\_breaks}] A numeric vector specifying manual numeric breaks.
Defaults to ggplot2::waiver(). See \code{\LinkA{scale\_y\_continuous}{scale.Rul.y.Rul.continuous}}.

\item[\code{y\_limits}] A numeric vector of length two providing limits for
the y-axis. Use NA to refer to the existing minimum or maximum.
Defaults to c(NA, NA). See \code{\LinkA{scale\_y\_continuous}{scale.Rul.y.Rul.continuous}}.

\item[\code{n\_y\_axis\_breaks}] An integer specifying the number of major breaks
for the x-axis. Defaults to NULL.
See \code{\LinkA{scale\_y\_continuous}{scale.Rul.y.Rul.continuous}}.

\item[\code{x\_axis\_text\_size}] A numeric value specifying the size of the
x-axis text. Defaults to 8. See \code{\LinkA{element\_text}{element.Rul.text}}.

\item[\code{y\_axis\_text\_size}] A numeric value specifying the size of the
y-axis text. Defaults to 12. See \code{\LinkA{element\_text}{element.Rul.text}}.

\item[\code{x\_axis\_title\_size}] A numeric value specifying the size of the
x-axis title. Defaults to 14. See \code{\LinkA{element\_text}{element.Rul.text}}.

\item[\code{y\_axis\_title\_size}] A numeric value specifying the size of the
y-axis title. Defaults to 14. See \code{\LinkA{element\_text}{element.Rul.text}}.

\item[\code{plot\_title\_size}] A numeric value specifying the size of the plot
title. Defaults to 20. See \code{\LinkA{element\_text}{element.Rul.text}}.

\item[\code{geom\_point\_size}] A numeric value specifying the size of the points
on the graph. Defaults to 2. See \code{\LinkA{geom\_point}{geom.Rul.point}}.

\item[\code{geom\_line\_width}] A numeric value specifying the width of the line
(applicable only for plot\_type = 2, 3, or 4). Defaults to 0.5.

\item[\code{ci\_plot\_annoate\_value}] A character string specifying whether to add
the doubling time or rate estimates from the model to plot 4. Options
include "double\_time" for the doubling time with 95\% CI, "rate" for the
rate estimate with 95\% CI, or "none" for no annotation. Defaults to
"double\_time"

\item[\code{annotate\_value\_text\_size}] A numeric value specifying the size of
the annotation text. Defaults to 6. See \code{\LinkA{geom\_text}{geom.Rul.text}}.
See \code{\LinkA{geom\_line}{geom.Rul.line}}.
\end{ldescription}
\end{Arguments}
%
\begin{Value}
Returns a ggplot2 plot
\end{Value}
%
\begin{SeeAlso}
\code{\LinkA{growth\_curve\_model\_fit}{growth.Rul.curve.Rul.model.Rul.fit}}
\end{SeeAlso}
%
\begin{Examples}
\begin{ExampleCode}
# Load example data (exponential data)
data(exp_mixed_data)
# Fit an mixed-effects growth model to the data
exp_mixed_model_summary <- growth_curve_model_fit(
  data_frame = exp_mixed_data,
  function_type = "exponential"
)
# Create growth vs time plot of data with fitted values (plot_type = 2)
exp_growth_plot <- growth_vs_time_plot(
  growth_model_summary_list = exp_mixed_model_summary,
  plot_type = 2
)
print(exp_growth_plot)
\end{ExampleCode}
\end{Examples}
\HeaderA{linear\_mixed\_model}{Fit a linear mixed-effects regression model}{linear.Rul.mixed.Rul.model}
%
\begin{Description}
'linear\_mixed\_model()' is a function utilized with the
\code{\LinkA{growth\_curve\_model\_fit}{growth.Rul.curve.Rul.model.Rul.fit}} function for fitting a linear
mixed-effects regression model to growth data utilizing the saemix
package
\end{Description}
%
\begin{Usage}
\begin{verbatim}
linear_mixed_model(
  data_frame,
  model_type = "mixed",
  fixed_rate = TRUE,
  num_chains = 1
)
\end{verbatim}
\end{Usage}
%
\begin{Arguments}
\begin{ldescription}
\item[\code{data\_frame}] A data frame object that at minimum contains three
variables:
\begin{itemize}

\item{} cluster - a character type variable used to specify how observations
are nested or grouped by a particular cluster. Note if using a
least-squares model, please fill in all values of cluster with a single
dummy character string, do NOT leave blank.
\item{} time - a numeric type variable used for measuring time such as
minutes, hours, or days
\item{} growth\_metric - a numeric type variable used for measuring growth
over time such as cell count or confluency

\end{itemize}


\item[\code{model\_type}] A character string specifying the type of regression
model to be used. If "mixed" a mixed-effects regression model will be used
with fixed and random effects to account for clustering. Defaults to "mixed".

\item[\code{fixed\_rate}] A logical value specifying whether the rate constant
of the function should be treated as a fixed effect (TRUE) or random
effect (FALSE). Defaults to TRUE

\item[\code{num\_chains}] A numeric value specifying the number of chains to run
in parallel in the MCMC algorithm of saemix. Defaults to 1.
\end{ldescription}
\end{Arguments}
%
\begin{Value}
Returns a linear model object of class 'saemix' when a mixed-effects
model is specified or a model object of class 'nls' if a least-squares
model is specified.
\end{Value}
%
\begin{SeeAlso}
\code{\LinkA{growth\_curve\_model\_fit}{growth.Rul.curve.Rul.model.Rul.fit}}
\end{SeeAlso}
%
\begin{Examples}
\begin{ExampleCode}
# Load example data (linear data from GrowthCurveME package)
data(lin_mixed_data)
# Fit a linear mixed-effects growth model
lin_mixed_model <- growth_curve_model_fit(
data_frame = lin_mixed_data,
function_type = "linear")
# Fit an linear mixed-effects model using linear_mixed_model()
lin_mixed_model <- linear_mixed_model(data_frame = lin_mixed_data)
\end{ExampleCode}
\end{Examples}
\HeaderA{lin\_mixed\_data}{Sample linear growth dataset}{lin.Rul.mixed.Rul.data}
\keyword{datasets}{lin\_mixed\_data}
%
\begin{Description}
A dataset containing the minimum required variables needed to input data
into the GrowthModelME package functions
\end{Description}
%
\begin{Usage}
\begin{verbatim}
lin_mixed_data
\end{verbatim}
\end{Usage}
%
\begin{Format}
A data frame with 110 rows and 3 variables:
\begin{description}

\item[cluster] A character type variable used to specify the clustering
of values by a particular metric. Note when selecting a least-squares
model instead of a mixed-effects model, do not leave this variable NA,
fill in this values for this variable with 1 repetitive dummy variable for
the package to run properly
\item[time] A numeric type variable for any measurement in time such as
minutes, hours, or days
\item[growth\_metric] A numeric type variable for measuring growth such as
confluency or cell count

\end{description}

\end{Format}
%
\begin{Source}
Created through simulation to serve as an example
\end{Source}
%
\begin{Examples}
\begin{ExampleCode}
data(log_mixed_data)
\end{ExampleCode}
\end{Examples}
\HeaderA{logistic\_mixed\_model}{Fit a logistic mixed-effects regression model}{logistic.Rul.mixed.Rul.model}
%
\begin{Description}
'logistic\_mixed\_model()' is a function utilized with the
\code{\LinkA{growth\_curve\_model\_fit}{growth.Rul.curve.Rul.model.Rul.fit}} function for fitting a logistic
mixed-effects regression model to growth data utilizing the saemix package.
Starting values are derived from an initial least-squares model using the
\code{\LinkA{nlsLM}{nlsLM}} function.
\end{Description}
%
\begin{Usage}
\begin{verbatim}
logistic_mixed_model(
  data_frame,
  model_type = "mixed",
  fixed_rate = TRUE,
  num_chains = 1
)
\end{verbatim}
\end{Usage}
%
\begin{Arguments}
\begin{ldescription}
\item[\code{data\_frame}] A data frame object that at minimum contains three
variables:
\begin{itemize}

\item{} cluster - a character type variable used to specify how observations
are nested or grouped by a particular cluster. Note if using a
least-squares model, please fill in all values of cluster with a single
dummy character string, do NOT leave blank.
\item{} time - a numeric type variable used for measuring time such as
minutes, hours, or days
\item{} growth\_metric - a numeric type variable used for measuring growth
over time such as cell count or confluency

\end{itemize}


\item[\code{model\_type}] A character string specifying the type of regression
model to be used. If "mixed" a mixed-effects regression model will be used
with fixed and random effects to account for clustering. Defaults to "mixed".

\item[\code{fixed\_rate}] A logical value specifying whether the rate constant
of the function should be treated as a fixed effect (TRUE) or random
effect (FALSE). Defaults to TRUE

\item[\code{num\_chains}] A numeric value specifying the number of chains to run
in parallel in the MCMC algorithm of saemix. Defaults to 1.
\end{ldescription}
\end{Arguments}
%
\begin{Value}
Returns a logistic model object of class 'saemix' when a
mixed-effects model is specified or a model object of class 'nls' if a
least-squares model is specified.
\end{Value}
%
\begin{SeeAlso}
\code{\LinkA{growth\_curve\_model\_fit}{growth.Rul.curve.Rul.model.Rul.fit}}
\end{SeeAlso}
%
\begin{Examples}
\begin{ExampleCode}
# Load example data (logistic data from GrowthCurveME package)
data(log_mixed_data)
# Fit a logistic mixed-effects growth model to the data
log_mixed_model <- growth_curve_model_fit(data_frame = log_mixed_data,
function_type = "logistic")
# Fit a logistic mixed-effected model using logistic_mixed_model()
log_mixed_model <- logistic_mixed_model(data_frame = log_mixed_data)
\end{ExampleCode}
\end{Examples}
\HeaderA{log\_mixed\_data}{Sample logistic growth dataset}{log.Rul.mixed.Rul.data}
\keyword{datasets}{log\_mixed\_data}
%
\begin{Description}
A dataset containing the minimum required variables needed to input data
into the GrowthModelME package functions
\end{Description}
%
\begin{Usage}
\begin{verbatim}
log_mixed_data
\end{verbatim}
\end{Usage}
%
\begin{Format}
A data frame with 320 rows and 3 variables:
\begin{description}

\item[cluster] A character type variable used to specify the clustering
of values by a particular metric. Note when selecting a least-squares
model instead of a mixed-effects model, do not leave this variable NA,
fill in this values for this variable with 1 repative dummy variable for
the package to run properly
\item[time] A numeric type variable for any measurement in time such as
minutes, hours, or days
\item[growth\_metric] A numeric type variable for measuring growth such as
confluency or cell count

\end{description}

\end{Format}
%
\begin{Source}
Created through simulation to serve as an example
\end{Source}
%
\begin{Examples}
\begin{ExampleCode}
data(log_mixed_data)
\end{ExampleCode}
\end{Examples}
\HeaderA{summarize\_growth\_model}{Summarize growth model object and data}{summarize.Rul.growth.Rul.model}
%
\begin{Description}
'summarize\_growth\_model()' is a function used to create a list object of
data frames based on a user's input data and outputed
growth model object from \code{\LinkA{growth\_curve\_model\_fit}{growth.Rul.curve.Rul.model.Rul.fit}}.
The list object (referred to in this package as 'growth\_model\_summary\_list')
can be used to extract model predicted values, residuals, and can be
in-putted into supporting functions from GrowthCurveME to
generate plots and perform model diagnostics.
\end{Description}
%
\begin{Usage}
\begin{verbatim}
summarize_growth_model(
  data_frame,
  growth_model_object,
  model_type = "mixed",
  function_type = "exponential",
  fixed_rate = TRUE,
  time_unit = "hours"
)
\end{verbatim}
\end{Usage}
%
\begin{Arguments}
\begin{ldescription}
\item[\code{data\_frame}] A data frame object that at minimum contains three
variables:
\begin{itemize}

\item{} cluster - a character type variable used to specify how observations
are nested or grouped by a particular cluster. Note if using a
least-squares model, please fill in all values of cluster with a single
dummy character string, do NOT leave blank.
\item{} time - a numeric type variable used for measuring time such as
minutes, hours, or days
\item{} growth\_metric - a numeric type variable used for measuring growth
over time such as cell count or confluency

\end{itemize}


\item[\code{growth\_model\_object}] The model object that is created using
the growth\_curve\_model\_fit()

\item[\code{model\_type}] A character string specifying the model\_type that was
fit using the \code{\LinkA{growth\_curve\_model\_fit}{growth.Rul.curve.Rul.model.Rul.fit}} function. Options
include either "mixed" or "least-squares. Defaults to "mixed".

\item[\code{function\_type}] A character string specifying the function for
modeling the shape of the growth. Options include "exponential", "linear",
"logistic", or "gompertz".

\item[\code{fixed\_rate}] A logical value specifying whether the rate constant
of the function should be treated as a fixed effect (TRUE) or random
effect (FALSE). Defaults to TRUE

\item[\code{time\_unit}] A character string specifying the units in which time is
measured in. Defaults to "hours"
\end{ldescription}
\end{Arguments}
%
\begin{Value}
A list object with the following data frames within the list:
\begin{itemize}

\item{} model\_summary\_wide - a data frame with 1 row containing
key model estimates, doubling-time, and model metrics depending
on the model\_type and function\_type specified
\item{} model\_summary\_long - a data frame that is a long dataset version of
'model\_summary\_wide' that can be used to generate a table of the model
results (see function \code{\LinkA{growth\_model\_summary\_table}{growth.Rul.model.Rul.summary.Rul.table}})
\item{} model\_residual\_data - a data frame containing the original data
frame values as well as predicted values, residuals, and theoretical
quantiles of the residuals depending on the model\_type selected
(see functions \code{\LinkA{growth\_model\_residual\_plots}{growth.Rul.model.Rul.residual.Rul.plots}} and
\code{\LinkA{growth\_vs\_time\_plot}{growth.Rul.vs.Rul.time.Rul.plot}}

\end{itemize}

\end{Value}
%
\begin{SeeAlso}
\code{\LinkA{growth\_curve\_model\_fit}{growth.Rul.curve.Rul.model.Rul.fit}}
\end{SeeAlso}
%
\begin{Examples}
\begin{ExampleCode}
# Load example data (exponential data)
data(exp_mixed_data)
# Fit an mixed-effects growth model to the data
exp_mixed_model <- growth_curve_model_fit(
data_frame = exp_mixed_data,
function_type = "exponential",
return_summary = FALSE)
# Summarize the data by creating a summary list object
exp_mixed_model_summary <- summarize_growth_model(
data_frame = exp_mixed_data,
growth_model_object = exp_mixed_model,
model_type = "mixed",
function_type = "exponential",
time_unit = "hours")
# Extracting a data frame from the list object
model_summary_wide <- exp_mixed_model_summary[["model_summary_wide"]]
\end{ExampleCode}
\end{Examples}
\HeaderA{summarize\_growth\_model\_ls}{Summarize least-squares growth model object and data}{summarize.Rul.growth.Rul.model.Rul.ls}
%
\begin{Description}
'summarize\_growth\_model\_mixed()' is a function used within the
\code{\LinkA{summarize\_growth\_model}{summarize.Rul.growth.Rul.model}} function to create a list object of
data frames based on a user's input data frame and outputed least-squares
growth model object from \code{\LinkA{growth\_curve\_model\_fit}{growth.Rul.curve.Rul.model.Rul.fit}}.
The list object (referred to in this package as 'growth\_model\_summary\_list')
can be used to extract model predicted values, residuals,
and can be in-putted into supporting functions from GrowthCurveME to
generate plots and perform model diagnostics.
\end{Description}
%
\begin{Usage}
\begin{verbatim}
summarize_growth_model_ls(
  data_frame,
  ls_model,
  function_type = "exponential",
  time_unit = "hours"
)
\end{verbatim}
\end{Usage}
%
\begin{Arguments}
\begin{ldescription}
\item[\code{data\_frame}] A data frame object that at minimum contains three
variables:
\begin{itemize}

\item{} cluster - a character type variable used to specify how observations
are nested or grouped by a particular cluster. Note if using a
least-squares model, please fill in all values of cluster with a single
dummy character string, do NOT leave blank.
\item{} time - a numeric type variable used for measuring time such as
minutes, hours, or days
\item{} growth\_metric - a numeric type variable used for measuring growth
over time such as cell count or confluency

\end{itemize}


\item[\code{ls\_model}] The least-squares model object that is created using
the 'growth\_curve\_model\_fit()'

\item[\code{function\_type}] A character string specifying the function for
modeling the shape of the growth. Options include "exponential", "linear",
"logistic", or "gompertz".

\item[\code{time\_unit}] A character string specifying the units in which time is
measured in. Defaults to "hours"
\end{ldescription}
\end{Arguments}
%
\begin{Value}
A list object with the following data frames within the list:
\begin{itemize}

\item{} model\_summary\_wide - a data frame with 1 row containing
key model estimates, doubling-time, and model metrics depending
on the model\_type and function\_type specified
\item{} model\_summary\_long - a data frame that is a long dataset version of
'model\_summary\_wide' that can be used to generate a table of the model
results (see function \code{\LinkA{growth\_model\_summary\_table}{growth.Rul.model.Rul.summary.Rul.table}})
\item{} model\_residual\_data - a data frame containing the original data
frame values as well as predicted values, residuals, and theoretical
quantiles of the residuals depending on the model\_type selected
(see functions \code{\LinkA{growth\_model\_residual\_plots}{growth.Rul.model.Rul.residual.Rul.plots}} and
\code{\LinkA{growth\_vs\_time\_plot}{growth.Rul.vs.Rul.time.Rul.plot}}

\end{itemize}

\end{Value}
%
\begin{SeeAlso}
\code{\LinkA{growth\_curve\_model\_fit}{growth.Rul.curve.Rul.model.Rul.fit}}
\code{\LinkA{summarize\_growth\_model}{summarize.Rul.growth.Rul.model}}
\end{SeeAlso}
%
\begin{Examples}
\begin{ExampleCode}
# Load example data (exponential data)
data(exp_mixed_data)
# Fit an mixed-effects growth model to the data
exp_ls_model <- growth_curve_model_fit(
data_frame = exp_mixed_data,
function_type = "exponential",
model_type = "least-squares",
return_summary = FALSE)
# Summarize the data by creating a summary list object
exp_ls_model_summary <- summarize_growth_model_ls(
data_frame = exp_mixed_data,
ls_model = exp_ls_model,
function_type = "exponential",
time_unit = "hours")
\end{ExampleCode}
\end{Examples}
\HeaderA{summarize\_growth\_model\_mixed}{Summarize mixed-effects growth model object and data}{summarize.Rul.growth.Rul.model.Rul.mixed}
%
\begin{Description}
'summarize\_growth\_model\_mixed()' is a function used within the
\code{\LinkA{summarize\_growth\_model}{summarize.Rul.growth.Rul.model}} function to create a list object of
data frames based on a user's input data frame and outputed mixed-effects
growth model object from \code{\LinkA{growth\_curve\_model\_fit}{growth.Rul.curve.Rul.model.Rul.fit}}.
The list object (referred to in this package as 'growth\_model\_summary\_list')
can be used to extract model predicted values, residuals, and can be
in-putted into supporting functions from GrowthCurveME to generate plots and
perform model diagnostics.
\end{Description}
%
\begin{Usage}
\begin{verbatim}
summarize_growth_model_mixed(
  data_frame,
  mixed_growth_model,
  function_type = "exponential",
  fixed_rate = TRUE,
  time_unit = "hours"
)
\end{verbatim}
\end{Usage}
%
\begin{Arguments}
\begin{ldescription}
\item[\code{data\_frame}] A data frame object that at minimum contains three
variables:
\begin{itemize}

\item{} cluster - a character type variable used to specify how observations
are nested or grouped by a particular cluster. Note if using a
least-squares model, please fill in all values of cluster with a single
dummy character string, do NOT leave blank.
\item{} time - a numeric type variable used for measuring time such as
minutes, hours, or days
\item{} growth\_metric - a numeric type variable used for measuring growth
over time such as cell count or confluency

\end{itemize}


\item[\code{mixed\_growth\_model}] The mixed-effects model object that is created
using the 'growth\_curve\_model\_fit()'

\item[\code{function\_type}] A character string specifying the function for
modeling the shape of the growth. Options include "exponential", "linear",
"logistic", or "gompertz".

\item[\code{fixed\_rate}] A logical value specifying whether the rate constant
of the function should be treated as a fixed effect (TRUE) or random
effect (FALSE). Defaults to TRUE

\item[\code{time\_unit}] A character string specifying the units in which time is
measured in. Defaults to "hours"
\end{ldescription}
\end{Arguments}
%
\begin{Value}
A list object with the following data frames within the list:
\begin{itemize}

\item{} model\_summary\_wide - a data frame with 1 row containing
key model estimates, doubling-time, and model metrics depending
on the model\_type and function\_type specified
\item{} model\_summary\_long - a data frame that is a long dataset version of
'model\_summary\_wide' that can be used to generate a table of the model
results (see function \code{\LinkA{growth\_model\_summary\_table}{growth.Rul.model.Rul.summary.Rul.table}})
\item{} model\_residual\_data - a data frame containing the original data
frame values as well as predicted values, residuals, and theoretical
quantiles of the residuals depending on the model\_type selected
(see functions \code{\LinkA{growth\_model\_residual\_plots}{growth.Rul.model.Rul.residual.Rul.plots}} and
\code{\LinkA{growth\_vs\_time\_plot}{growth.Rul.vs.Rul.time.Rul.plot}}

\end{itemize}

\end{Value}
%
\begin{SeeAlso}
\code{\LinkA{growth\_curve\_model\_fit}{growth.Rul.curve.Rul.model.Rul.fit}}
\code{\LinkA{summarize\_growth\_model}{summarize.Rul.growth.Rul.model}}
\end{SeeAlso}
%
\begin{Examples}
\begin{ExampleCode}
# Load example data (exponential data)
data(exp_mixed_data)
# Fit an mixed-effects growth model to the data
exp_mixed_model <- growth_curve_model_fit(
data_frame = exp_mixed_data,
function_type = "exponential",
return_summary = FALSE)
# Summarize the data by creating a summary list object
exp_mixed_model_summary <- summarize_growth_model_mixed(
data_frame = exp_mixed_data,
mixed_growth_model = exp_mixed_model,
fixed_rate = TRUE,
function_type = "exponential",
time_unit = "hours")
model_summary_wide <- exp_mixed_model_summary[["model_summary_wide"]]
\end{ExampleCode}
\end{Examples}
\printindex{}
\end{document}
